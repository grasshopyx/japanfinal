\documentclass[utf8]{ctexart}
\setCJKmainfont[ItalicFont={KaiTi},BoldFont={SimHei}]{SimSun}
\setCJKsansfont{SimHei}
\setCJKmonofont{FangSong}
\usepackage{graphicx}
%\usepackage{listings}	源代码
\usepackage{booktabs}
\title{日本武士的兴盛和灭亡探因}
\author{计算机应用技术专业,李颂元, 21321293}
\bibliographystyle{plain}
\begin{document}
\maketitle
\section*{摘要}
本文介绍日本武士、武士制度和武士道,试图通过文化比较来探求武士的兴盛和灭亡的原因。本文认为武士的兴盛源于封建制和军政体制,武士的衰亡并不是因为简单的兵器落后,而是因为武士制度已无法适应近现代的发展。

\textbf{关键词:武士、武士制度、武士道、中国武士、骑士}
\tableofcontents
\section{日本武士简介}
\subsection{日本武士}
日本武士,日语中通常写作“侍”,是一种职业,一种制度,一种文化。职业上就叫武士,制度上称为武士制度,文化上则为武士道。

日本武士首先是一种职业,是日本从平安时期兴起到明治时期初期存在的一种职业军人。他们效忠于主公(主公本身也是武士),而非天皇。他们并不从事生产,战时打仗,平时修习武艺,是精锐的军事力量。

武士制度则是对武士阶层的政治保障。武士制度在幕府时代较为典型。幕府制度建立起来之后,武士集团掌握着政权,武士实际上就是统治阶层,地位在平民之上,地位世袭,拥有对平民格杀勿论的特权\cite{}。

武士道则是武士的价值观念和行为准则。武士并不只会武艺,只会武艺的人可以打天下,不能治天下。作为统治阶层,武士必须是文武全才,是社会的精英,武士道也是在这种背景下产生的。

从上面的论述可以看出,武士既是军事力量,又是政治力量。在日本的武士时期,武士是社会的中心,由一整套武士制度保障他们的统治,同时他们还有一套文化,即武士道,来自我完善。
\subsection{日本武士和中国武士的比较}
中国也有武士这个词汇,但是意思不太一样。在中国古代,武士可以指携带武器的士兵,也可以指春秋战国时期的一个特殊阶层,代表人物有荆轲。这两个定义中,后者比较接近日本武士,但是还是有很大区别。我们来讨论一下后者。

中国春秋战国时期的武士和日本武士的起源也是类似的,都是地方豪强养的一批武艺高强的人,是中国的春秋时期叫“门客”的一种。可见中国武士和日本武士一样,都不是国家招募的军人。但是,中国武士就停留在日本武士的初期阶段,没有继续发展下去,没有形成武士制度,更不用说武士道。从中日历史的对比来看,恐怕是因为中国在战国时代结束后的秦朝是一个中央集权的王朝,结束了传统意义的“封建制”(封土建国),后面的王朝也延续了这个传统。中央集权国家职业军人是通过国家征兵产生,也就没有了形成日本那种武士的土壤。而日本那边,则是传统意义上的“封建制”,并且在幕府时代最为典型,而在江户时代确立的幕藩体制则将这个制度发展到了全盛。

总的来说,日本武士和中国武士是非常不一样的东西。日本武士和中国武士在萌芽时期表现比较相似,但是由于历史的原因,日本武士发展了下去,并且兴盛,中国武士则退出了历史舞台。
\subsection{日本武士与欧洲骑士的比较}
欧洲骑士和日本武士比较相似。欧洲骑士是一种职业,也有一套骑士制度,也有一套文化,即骑士精神。

欧洲骑士的兴盛,我认为和日本武士原因是类似的,都是因为长时间经历着“封建制”。在封建制下,领主大规模征兵并不现实,他们需要一批精锐的力量来保卫自己的领地和吞并其他领主的领地。

但是,欧洲骑士和日本武士又有所不同。骑士并不掌握着政权,社会地位没有日本武士那么高,往往只有一小块封地,地位并不世袭\cite{}。总的来说,差异的核心是政治地位的差异。造成这个差异的原因恐怕是欧洲的宗教势力在政治中所扮演的特殊角色。自从罗马帝国把基督教确立为国教,欧洲社会就是政教合一的。正是由于基督教势力在政治中的特殊地位,决定了骑士不会在统治阶层的顶峰。日本武士时期并不是没有宗教,但是日本那时的宗教主要是汉传佛教,而汉传佛教的教义与政治无关。所以,即使佛教再兴盛,也不会动摇武士统治的根基。甚至,许多武士都信仰佛教,因为武士道与佛教教义有一致的地方。

\section{武士的兴盛原因}
前文已经提到,武士起源于地方豪强招募的“门客”。中国武士还没有真正兴起,就在中央集权王朝的到来而变成了历史。欧洲的骑士与日本武士相似,却在宗教背景下,不能产生军人政权。综合上述观点,我们就可以得到武士兴起的原因。

日本武士的兴盛需要封建制。这里的封建制依然指的是封土建国。在封建制下,领主为了保卫自己的庄园,招募精锐的军事力量。武士应运而生。

日本武士的兴盛需要武士拥有最高的政治地位,即掌握政权。在军政体制下,武士的利益得到最大的保障和发展的空间。

\section{武士的灭亡原因}
直观上看,武士灭亡的原因非常的简单:武士在近现代的热兵器面前毫无还手之力。美国电影《最后的武士》中,最后的武士倒在了机关枪的枪口之下。日本电影《七武士》则有更多的暗语,例如七武士最忌惮山贼的是他们手中的火枪,武士不得已要去抢山贼手中的火枪,死去的四个武士都是死于火枪等等。但实际上,事情并没有那么简单。

前文已经提到,武士首先是职业军人,军人用什么武器,应该随着时代的发展而有所改变。在热兵器时代,还用冷兵器作为武器,那肯定是自取灭亡。所以武士完全可以与时俱进,把冷兵器换成热兵器,其余的不变,这样,武士就保全下来了。

如果近代化是日本开启的,那么武士可能真的就被这么保全下来了。但是,日本的近代化是被黑船打开的。日本开国之后,又经历了大政奉还、明治维新等历史时间。日本是在学习西方制度的过程中逐步废除了武士制度。是因为武士制度禁锢了现代化,所以要废除武士制度。四民平等,废刀令等政策让传统武士在制度上失去了政治保障,武士制度最终瓦解。

武士制度虽然终结了,武士道精神却仍在继续。只不过武士道的忠于主公的价值观念,被明治时期逐步修正为忠于天皇。日本的近代军人拿着热兵器,继续在战场上表现出了武士道的精神。例如肉搏战时退弹拼刺刀,战败后剖腹谢罪等等。这实际上也说明了,兵器的演化并不是武士终结的直接原因。

日本武士真正终结于日本战败。随着日本战败,军国主义在日本彻底破产。日本和平宪法规定日本不得拥有军队,仅保留自卫队。日本军政体制再也没有出现,也就不会有武士。

综上所述,兵器的演化并不是武士破产的原因,武士不适应近现代的文明,才是武士灭亡的原因。武士的灭亡分两步,第一步是明治维新,取消了武士这种世袭职业军人,废除了武士制度,但是武士道还在;第二步是日本战败,军国主义破产,武士彻底退出历史舞台。

\section{总结}
日本武士是日本特有的一种现象,它是职业、制度与文化的统一。武士的起源在各国是相似的,中国、日本、欧洲都是如此。但是武士的兴盛需要土壤,封建制和军政体制是日本武士兴盛的原因。日本武士并不是因为兵器落后而走向衰亡,而是分两步,即明治维新和日本战败,由于不适应近现代社会而灭亡。

\bibliography{japan}


\newpage
\end{document}
\documentclass[utf8]{ctexart}
\setCJKmainfont[ItalicFont={KaiTi},BoldFont={SimHei}]{SimSun}
\setCJKsansfont{SimHei}
\setCJKmonofont{FangSong}
\usepackage{graphicx}
\usepackage{listings}
\usepackage{booktabs}
\title{日本武士制度的兴起和灭亡}
\author{李颂元, 21321293}
\begin{document}
\maketitle
\tableofcontents
%\section*{写在前面}
%
%\section{Introduction}
%
%\subsection{分析}
%这个Introduction值得反复看几遍,因为它归纳,概括了后面几章的内容。这里以列表的形式把本章的一些有用的内容列出来。
%\subsubsection{一些操作系统使用ELF的时间}
%\begin{table}[htbp]
%	\caption{一些操作系统使用ELF的时间}
%	\centering
%	\begin{tabular}{cc}
%	\toprule[1.5pt]
%	OS	&	Time\\
%	\midrule[1pt]
%	RHL2.0 Beta & late summer 1995\\
%	Slackware v3.0 & November 1995\\
%	NetBSD & January 1998\\
%	FreeBSD 3.0 & October 1998\\
%	\bottomrule[1.5pt]
%	\end{tabular}
%\end{table}
%\subsubsection{本书的内容}
%这本书会覆盖以下内容。
%\begin{enumerate}
%\item ELF格式基础。ELF如何构建可执行的、可重定位的、共享的对象内容。
%\item 如何开始建造使用libelf的应用程序。
%\item ELF(3)和GELF(3)所提供的抽象:如何抽象ELF类(32、64)和大小端。
%\item 使用API以观察ELF的内部结构。
%\item 使用ELF库创建一个新的ELF object。
%\item 介绍类无关的GELF,何时何地使用GELF。
%\item 如何处理ar文档。
%\end{enumerate}
%\subsubsection{每章概要}
%\begin{table}[htbp]
%	\caption{每章概要}
%	\centering
%\begin{tabular}{cl}
%	\toprule[1.5pt]
%	章	&	概要\\
%	\midrule[1pt]
%	Ch02	& 起步	\\
%	Ch03	& 查看ELF的内部结构\\
%	Ch04	& Ehdr(Executable program header table)\\
%	Ch05	&数据如何存储在section里\\
%	Ch06	&如何创建ELF object\\
%	Ch07	&如何处理ar文档\\
%	Ch08	&更进一步的文献\\
%	\bottomrule[1.5pt]
%\end{tabular}
%\end{table}
%\subsubsection{每章介绍的库函数}
%\begin{table}[htbp]
%\caption{每章介绍的库函数}
%\begin{tabular}{cl}
%\toprule[1.5pt]
%章	&	函数名\\
%\midrule[1pt]
%Ch02	&	elf\_begin, elf\_end, elf\_errmsg, elf\_errno, elf\_kind and elf\_version\\
%Ch03	&	elf\_getident, elf\_getphdrnum, elf\_getshdrnum, \\
%		&	elf\_getshdrstrndx, gelf\_getehdr and gelf\_getclass\\
%Ch04	&	gelf\_getphdr\\
%Ch05	&	elf\_getscn, elf\_getdata, elf\_nextscn, elf\_strptr, and gelf\_getshdr\\
%Ch07	&	elf\_getarhdr, elf\_getarsym, elf\_next and elf\_rand\\
%\bottomrule[1.5pt]
%\end{tabular}
%\end{table}



\newpage
\end{document}